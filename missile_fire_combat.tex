\section{MISSILE FIRE COMBAT}
\hfill

GENERAL RULE:

Any unit with missile capability (see Sample Units, 2.2) can fire in a friendly fire segment. Such units can fire at any enemy units within the range allowed on the Missile Fire Matrix. Bowmen units (only) can use \textit{high trajectory fire} to shoot over friendly units or enemy units. There are restrictions on the ability to fire, as well as the number of times a missile unit can fire.

PROCEDURE:

During a friendly missile fire segment, the firing player determines which of his eligible missile units will fire at which enemy units. Using the Missile Fire Matrix, he cross references the unit type (bowman, javelin or sling) with the range in hexes from the enemy target. The resultant number is the \textit{fire strength} of the firing unit for that attack.

The player totals the fire strengths of all units firing at a given enemy unit and compares the total to the enemy unit's \textit{defensive strength} (which depends on the battle order in effect). Adjustments for terrain and high trajectory are taken, and the comparison is stated as a simple \textit{odds ratio} of attacker to defender. All odds are rounded off in favor of the defender; thus 10 to 3 is reduced to 3-1.

The firing player rolls one die and cross-references the die roll with the odds ratio on the Missile Fire Combat Results Table. Any result is applied immediately.

\subsection{How to Engage in Missile Combat}

\subsubsection[Missile Capability]{} Only units with a \textit{missile capability letter} can attack with missiles (see sample units, 2.2).

\subsubsection[Missile Strength]{} To find a unit's fire strength, the player counts the number of hexes to the target unit, excluding the firing unit's hex but including the target unit's hex. He then cross-references this range with the type of unit firing (on the Missile Fire Matrix, see map) to get that unit's fire strength. Note that the maximum ranges for each type of missile unit are limited.

\subsubsection[Combined Fire]{} Missile units can combine their strengths when firing at a given target, the strength contributed by each depending on range. An individual unit cannot split its fire strength between two or more hexes.

\subsubsection[Missile Procedure]{} After determining the total fire strength, missile fire proceeds as stated in the Procedure, above.

\subsubsection[Melee and Missile]{} Units can engage in missile fire the same turn in which they melee.

\subsubsection[Missile Fire Results]{} The Missile Combat Results Table (see map) has three results:

\begin{itemize}
  \item \textbf{M = Morale Check}. The affected unit undergoes an immediate \textit{morale check} (see 9.6).
  \item \textbf{D = Disrupted}. Place a Disrupted Marker on the unit to indicate its disrupted status (see 9.22).
  \item \textbf{1 = Reduced}. The affected unit is flipped over to indicate its reduced status. If already reduced, it is eliminated (see 9.24).
\end{itemize}

\subsubsection[Leaders]{} If a leader is stacked with a unit that suffers a "1" result, the owning player consults the Leader Casualty Table (10.55) for possible leader loss. A leader receiving missile fire when alone in a hex has a nominal defensive strength of "1".

\subsection{Missile Fire Restrictions}

\subsubsection[When Missiles Fire]{} Missile fire occurs only in the missile fire segments. Missile units can fire while in any battle order.

\subsubsection[Direction]{} Missile units can fire only through their frontal hexsides. The can fire at an enemy unit through any facing that that enemy unit presents. See 6.23 for the effects of flank and rear fire.

\subsubsection[ZOC Restrictions]{} A missile unit can fire at any enemy within range (7.12). However, if a missile unit is in the ZOC of an enemy unit it \textit{must} fire at one of the enemy units exerting such ZOC if it fires at all; it cannot fire elsewhere, even if it cannot fire at the enemy unit exerting the ZOC.

\subsubsection[Woods]{} Missile units can fire into woods hexes, but not \textit{through} them. Furthermore, missile units cannot fire into a hex if a hex or hexside of \textit{higher} elevation lies between the hexes (e.g. a unit in 0708 cannot fire at a unit in 0909 because the higher elevation in 0808 blocks the fire). Use a straightedge from the center of the firing unit's hex to the center of the target unit's hex to determine if terrain blocks fire. If the line lies exactly along a hexside between blocking and non-blocking terrain, fire is not blocked.

\subsubsection[Firing Through Units]{} Units cannot fire through other combat units, friendly or enemy, unless they use \textit{high trajectory} fire (7.3).

\subsubsection[Leaders]{} Leaders have no effect on missile fire combat.

\subsubsection[Low Odds]{} Missile fire at less than 1-4 odds has \textbf{no effect}. Fire at greater than 5-1 odds uses the 5-1 column.

\subsection{High Trajectory Fire}

\subsubsection[Low Trajectory Fire]{} All normal missile fire is considered low trajectory fire, i.e. the firing troops aim directly at their target. When the firing units shoot into the air, letting their shots fall at a steep angle into a target, this is called \textit{high trajectory fire}. Only bowmen can use high trajectory fire.

\subsubsection[Saxon Bowmen]{} Saxon bowmen can use such fire any time; Norman bowmen cannot use it until the Second Assault Period.

\subsubsection[Resolution]{} High trajectory fire is resolved as normal fire, except that all results on the Missile Fire CRT are read one column to the \textit{left}. Thus high trajectory fire at 2-1 odds would be resolved on the 1.5-1 column.

\subsubsection[Terrain]{} High trajectory fire allows a bowman unit to fire over friendly or enemy units. It does \textit{not} allow fire through woods or higher terrain hexes.

\subsection{Missile Supply Limits}

COMMENT:

Because of the lack of Saxon bowmen and Norman spearmen, the opposing sides could not use enemy shafts to replenish their own. The players are faced with a dwindling supply of missiles as the battle progresses.

\subsubsection[Fire Segment]{} Norman bowmen and Saxon javelin/spear units can fire only in a specific number of fire segments within a given Assault Period, as listed on the Missile Supply Table (7.45; see also 7.44).

\subsubsection[One and All]{} If one friendly unit fires in a given segment, all friendly units of the same wing or section are considered to have fired in that segment for the purpose of missile supply. \textit{(Optional Rule: Players who like paperwork can keep track of units individually. Good luck.)}

\subsubsection[Saving Ammo]{} If missile units in a given wing/section fire fewer times than those allotted for that Assault Period, they can transfer the unused segments to the next Assault Period. However, no unit can ever fire in more than \textit{six} segments in a given period.

\subsubsection[Saxon Bowmen]{} Saxon bowmen and slingers have no supply problems and can fire in any number of segments.

\subsection{Missile Supply Table}

\begin{tabular}{ |l|cc| }
  \hline & & \\[-2.0ex]
  \textbf{Assault} & \multicolumn{2}{c|}{\textbf{Fire Segments Allowed}} \\
  \textbf{Period} & Norman Bowmen & Saxon Jav/Sps \\
  \hline & &\\ [-2.0ex]
  First & 6 & 4 \\
  Second & 3 & 2 \\
  \hline
\end{tabular}

\subsection{The Missile Combat Tables}
\subsection{Missile Fire Matrix}

{\def\arraystretch{1.2}
  \begin{tabular}{ |lcccc| }
    \hline
    & & \multicolumn{3}{c|}{ \textbf{Range}} \\
    \hline
    Weapon & Max. Range & 3 & 2 & 1* \\
    \hline
    Bow & 3 hexes & 3 & 4 & 5 \\
    Javelin & 2 hexes & - & 2 & 3 \\
    Sling & 3 hexes & 1 & 1 & 2 \\
    \hline
  \end{tabular}
}
\par
The resultant number is the missile fire strength of the unit.

* Bowmen cannot enter ZOC. Bowmen starting a movement segment in an enemy ZOC must move out if possible.

\subsection{Missile Combat Results Table}

{\def\arraystretch{1.2}
\resizebox{\columnwidth}{!}{\begin{tabular}{ |ccccccccccc| }
  \multicolumn{11}{c}{ \textbf{Fire Ratio} } \\
  \hline& & & & & & & & & &\\[-2.5ex]
  Die Roll & 1-4 & 1-3 & 1-2 & 1-1.5 & 1-1 & 1.5-1 & 2-1 & 3-1 & 4-1 & 5-1 \\
  \hline
  1 & -- & -- & -- & -- & -- & M & M & M & D & D \\
  \rowcolor{BabyBlue}
  2 & -- & -- & -- & -- & M & M & M & D & D & D \\
  3 & -- & -- & -- & M & M & M & D & D & D & 1 \\
  \rowcolor{BabyBlue}
  4 & -- & -- & M & M & M & D & D & D & 1 & 1 \\
  5 & -- & M & M & M & D & D & D & 1 & 1 & 1 \\
  \rowcolor{BabyBlue}
  6 & M & M & M & D & D & D & 1 & 1 & 1 & 1 \\
  \hline
\end{tabular}}
}
\par
Attacks at less than 1-4 are ineffective. Attacks at greater than 5-1 odds are treated as 5-1.

Explanation of results: "--" = No effect; M = Unit must undergo a morale check; D = Unit is disrupted; 1 = Unit loses 1 step and any leader present must be checked for possible loss.
