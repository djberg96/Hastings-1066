\section{MELEE RESULTS}
\hfill

\subsection{Melee Results Table}

{\def\arraystretch{1.4}
\resizebox{\columnwidth}{!}{
\begin{tabular}{ |cccccccccccc| }
  \multicolumn{12}{c}{ \textbf{Melee Differential} } \\
  \hline& & & & & & & & & & & \\[-2.5ex]
  Die Roll & -6 & -5 & -4 & -3 & -2 & -1 & 0 & +1 & +2/+3 & +4/+5 & +6 \\
  \hline& & & & & & & & & & & \\[-2.5ex]
  1 & 1/- & 1/- & 1/- & 1/- & 1/- & 1/D* & 1/D* & M/- & D*/- & 1/1 & D*/1 \\
  \rowcolor{BabyBlue}
  2 & 1/- & 1/- & 1/- & 1/- & 1/D* & M/- & M/- & D*/- & 1/1 & D*/1 & -/M \\
  3 & 1/- & 1/- & 1/- & 1/D* & M/- & D*/- & 1/1 & D*/1 & -/M & -/1 & -/1\\
  \rowcolor{BabyBlue}
  4 & 1/- & 1/- & 1/D* & M/1 & D*/- & 1/1 & 1/1 & D*/1 & -/M & -/1 & -/1\\
  5 & 1/- & 1/D* & M/- & D*/- & 1/1 & 1/1 & D*/1 & -/M & -/1 & -/1 & -/1M\\
  \rowcolor{BabyBlue}
  6 & 1/D* & M/- & D*/- & 1/1 & 1/1 & D*/1 & -/M & -/1 & -/1 & -/1M & -/1M\\
  \hline
\end{tabular}
}}

\subsection{Explanation of Melee Results}

\subsubsection[Morale Check]{M = Morale Check.} The affected unit undergoes an immediate morale check (see 9.6).

\subsubsection[Disrupted]{D = Disrupted.} Place a Disrupted Marker on the unit to indicate its disrupted status. The following rules apply to disrupted units:

\begin{enumerate}
  \item Disrupted units cannot move with one exception: a disrupted unit in an enemy ZOC with an order that allows movement can move one hex out of that enemy ZOC in a friendly reaction segment (only). At the completion of such movement, the unit must check morale as per 5.45.
  \item Disrupted units do not have a ZOC and cannot attack by fire or melee. Such units defend against melee normally, using the current order. Further disruption have no additional effect.
  \item Disrupted units suffering a \textbf{reduced} result are reduced and remain disrupted.
  \item Disrupted units suffering a \textbf{rout} result have their status changed to routed.
  \item If the attacker is disrupted when attacking with more than one unit, \textit{all} attacking units are disrupted.
    \item If the attacker is in Charge or Attack \& Pursue order, defending units receiving a disruption result are routed instead.
  \item Disruption results can be removed by rally check die rolls conducted in the rally segment (9.5).
\end{enumerate}

\subsubsection[Routed]{R = Routed.} Routed units retreat three hexes (not movement points) as soon as the rout result is received. Saxons rout toward the north map edge, Normans toward the south. Place a Routed marker on the unit. The following rules apply to routed units:

\begin{enumerate}
  \item A routed unit retreats along the path of least resistance. A unit unable to move because of enemy units or ZOCs is reduced and disrupted instead (see displacement, 9.3).
  \item Routed units must face their rear lines whenever possible. For the Saxon units, this is north; for the Norman forces, south. Routed units always move toward their rear lines if possible.
  \item If a routed unit moves through a friendly unit, the friendly unit must make an immediate morale check.
  \item Routing units that leave the map are eliminated.
  \item Routed units have no ZOC and cannot attack by fire or melee. They cannot use their Shield Wall defensive strength, but use their normal defensive strength instead.
  \item Routed units that fail to rally in the rally segment must move two hexes to the rear. The unit must immediately stop upon entering an enemy ZOC. Routed units cannot move during the movement segment.
  \item Routed units ignore \textbf{"D"} results. Additional \textbf{"R"} results force them to retreat an additional three hexes.
\end{enumerate}

\subsubsection[Reduced]{1 = Reduced.} The affected unit is flipped over to its reduced side. If already reduced, it is eliminated. \textit{Being reduced has no effect on a unit's capabilities}.

\begin{enumerate}
  \item If currently routed, the unit is reduced and remains routed.
  \item If the attacker is reduced when attacking with more than one unit, only one attacking unit is reduced.
\end{enumerate}

\subsection{Retreats and Displacement}

\subsubsection[Owning Player]{} All retreats are conducted by the owning player. Retreating units must take the path of least resistance in terms of both terrain and occupied vs unoccupied hexes.

\subsubsection[Routed Units]{} Routed units can retreat through, but not stop in, friendly-occupied hexes. Any unit whose hex is routed through by a friendly unit must undergo an immediate morale check.

\subsubsection[Blocked Retreat]{} Routed units that cannot complete their retreat because of friendly units can voluntarily \textit{displace} the blocking unit by moving it one hex in any direction. A friendly unit cannot be displaced into a enemy ZOC or hex. The displaced unit is immediately disrupted.

\subsubsection[Displacement]{} Any number of friendly units can be displaced, in a chain reaction. However, if any unit in the chain cannot be displaced for any reason, no displacement at all takes place, and the original retreating unit is disrupted and reduced in the last unblocked hex it could enter.

\subsection{Pursuit}

\subsubsection[No Advance]{} Victorious units cannot move into a hex vacated by an enemy unit unless eligible to conduct pursuit.

\subsubsection[When Pursuit Occurs]{} Pursuit occurs whenever a unit, whether attacking or defending, is in Charge or Attack \& Pursue order and routs an enemy unit. The victorious unit, if undisrupted, \textit{must} pursue the enemy unit. If disrupted, no pursuit is allowed. Knights disrupted after charging cannot pursue.

\subsubsection[After Rout]{} Pursuit occurs only after a rout, and only in the battle orders noted above. It does not occur at any other time.

\subsubsection[When Pursuit Stops]{} Immediately after the routed unit is retreated three hexes, the victorious Charging or Attacking \& Pursuing units \textit{must} move after the routed unit. They will stop only when they either move adjacent to the routed unit or into an enemy ZOC. This advance is not considered normal movement and does not expend movement points.

\subsubsection[Enemy ZOCs]{} While pursuing, friendly units can ignore the ZOC of enemy units they start in; however, they must stop as soon as they enter a new enemy ZOC.

\subsubsection[Followup Movement]{} Units that have pursued can still move normally (according to their battle) in the ensuing friendly movement segment.

\subsection{Rally}

\subsubsection[When Rally Happens]{} Routed and disrupted units must be rallied to function normally. Rally attempts take place in the rally segment.

\subsubsection[How to Rally]{} To rally a disrupted unit, the player notes the unit's morale, rolls one die and consults the Rally Table (9.54). For example, a "C" class unit requires a roll of 1, 2 or 3 to rally. On a roll of 4, 5 or 6, the unit remains disrupted.

A unit can attempt to rally only once per segment. If there is a leader within rally range (10.1) of the unit, subtract 1 from the die roll.

\subsubsection[Routed Units]{} Routed units are automatically rallied if they are within the rally range of a friendly leader in the rally segment. \textit{No die roll is necessary}. If a routed unit is not rallied, it retreats two hexes, following the retreat pattern described in 9.23.

\begin{hangsubsubsection}
  \subsubsection{The Rally Check Table}
\end{hangsubsubsection}

\begin{tabular}{ |cc| }
    \hline & \\[-2.0ex]
    \textbf{Morale Rating} & {\textbf{Die Roll}} \\
    \hline & \\ [-2.0ex]
    A & 1-5 \\
    B & 1-4 \\
    C & 1-3 \\
    D & 1-2 \\
    E & 1 \\
    \hline
\end{tabular}

A successful rally check removes a disruption marker. Rout markers are automatically removed if the routed unit is within the rally range of a friendly leader in the Rally phase.

\subsection{Morale Checks}

\subsubsection[Conducting Morale Checks]{} To conduct a morale check for a unit, the player notes the unit's morale, rolls one die and consults the Morale Table (9.7). For example, a "C" class unit is disrupted on a roll of 4 and routed on a roll of 5 or 6. On a roll of 1, 2 or 3, the unit suffers no effect.

\subsubsection[Cavalry]{} Cavalry is subject to disorganization by a variety of factors that, while not strictly morale considerations, have been integrated into the morale system. Thus, knights must check morale under the following circumstances:

\begin{enumerate}
  \item Upon entering a marsh hex; or
  \item Before attempting to cross a ridge hexside (5.35), even during reaction; or
  \item After engaging in melee in Charge order.
\end{enumerate}

In all the above cases, any rout results are treated as disruptions.

\subsubsection[Reaction Movement]{} Units using reaction movement (5.4) must check morale after completing the movement.

\subsubsection[Adjacent Units]{} Units adjacent to friendly units routed in melee by opponents in Charge or Attack \& Pursue order must check morale. Disruption results are \textit{ignored}; rout results apply. Case 9.62 has precedence over this case, however.

\subsubsection[No Limits]{} There is no limit to the number of times a given unit can be forced to check morale in a segment; a check is made whenever required by the game situation.

\subsection{Morale Check Table}

\begin{tabular}{ |cccccc| }
    \multicolumn{6}{c}{ \textbf{Morale Level}} \\
    \hline & & & & & \\[-2.0ex]
    Die Roll & A & B & C & D & E \\
    \hline \\ [-2.0ex]
    1 & -- & -- & -- & -- & -- \\
    2 & -- & -- & -- & D & D \\
    3 & -- & -- & -- & D & R \\
    4 & -- & D & D & R & R \\
    5 & D & D & R & R & R \\
    6 & D & R & R & R & R \\
    \hline
\end{tabular}