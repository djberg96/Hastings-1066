\section{MELEE RESULTS}
\hfill

\subsection{Melee Results Table} (see map)

\subsection{Explanation of Melee Results}

\subsubsection[Morale Check]{M = Morale Check.} The affected unit undergoes an immediate morale check (see 9.6).

\subsubsection[Disrupted]{D = Disrupted.} Place a Disrupted Marker on the unit to indicate its disrupted status. The following rules apply to disrupted units:

\begin{enumerate}
  \item Disrupted units cannot move with one exception: a disrupted unit in an enemy ZOC with an order that allows movement can move one hex out of that enemy ZOC in a friendly reaction segment (only). At the completion of such movement, the unit must check morale as per 5.45.
  \item Disrupted units do not have a ZOC and cannot attack by fire or melee. Such units defend against melee normally, using the current order. Further disruption have no additional effect.
  \item Disrupted units suffering a \textbf{reduced} result are reduced and remain disrupted.
  \item Disrupted units suffering a \textbf{rout} result have their status changed to routed.
  \item If the attacker is disrupted when attacking with more than one unit, \textit{all} attacking units are disrupted.
    \item If the attacker is in Charge or Attack \& Pursue order, defending units receiving a disruption result are routed instead.
  \item Disruption results can be removed by rally check die rolls conducted in the rally segment (9.5).
\end{enumerate}

\subsubsection[Routed]{R = Routed.} Routed units retreat three hexes (not movement points) as soon as the rout result is received. Saxons rout toward the north map edge, Normans toward the south. Place a Routed marker on the unit. The following rules apply to routed units:

\begin{enumerate}
  \item A routed unit retreats along the path of least resistance. A unit unable to move because of enemy units or ZOCs is reduced and disrupted instead (see displacement, 9.3).
  \item Routed units must face their rear lines whenever possible. For the Saxon units, this is north; for the Norman forces, south. Routed units always move toward their rear lines if possible.
  \item If a routed unit moves through a friendly unit, the friendly unit must make an immediate morale check.
  \item Routing units that leave the map are eliminated.
  \item Routed units have no ZOC and cannot attack by fire or melee. They cannot use their Shield Wall defensive strength, but use their normal defensive strength instead.
  \item Routed units that fail to rally in the rally segment must move two hexes to the rear. The unit must immediately stop upon entering an enemy ZOC. Routed units cannot move during the movement segment.
  \item Routed units ignore \textbf{"D"} results. Additional \textbf{"R"} results force them to retreat an additional three hexes.
\end{enumerate}

\subsubsection[Reduced]{1 = Reduced.} The affected unit is flipped over to its reduced side. If already reduced, it is eliminated. \textit{Being reduced has no effect on a unit's capabilities}.

\begin{enumerate}
  \item If currently routed, the unit is reduced and remains routed.
  \item If the attacker is reduced when attacking with more than one unit, only one attacking unit is reduced.
\end{enumerate}

\subsection{Retreats and Displacement}

\subsubsection[Owning Player] All retreats are conducted by the owning player. Retreating units must take the path of least resistance in terms of both terrain and occupied vs unoccupied hexes.

\subsubsection[Routed Units]{} Routed units can retreat through, but not stop in, friendly-occupied hexes. Any unit whose hex is routed through by a friendly unit must undergo an immediate morale check.

\subsubsection[Blocked Retreat]{} Routed units that cannot complete their retreat because of friendly units can voluntarily \textit{displace} the blocking unit by moving it one hex in any direction. A friendly unit cannot be displaced into a enemy ZOC or hex. The displaced unit is immediately disrupted.

\subsubsection[Displacement]{}