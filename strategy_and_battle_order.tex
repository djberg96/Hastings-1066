\section{STRATEGY AND BATTLE ORDER}
\hfill

GENERAL RULE:

In the Order Phase of every Battle Turn, each player selects a basic strategy for each wing or nationality in his army for the ensuing Battle Turn. The Order Tables simulate the general inefficiency and lack of control that plagued medieval commanders.

\subsection{Wings and Nationalities}

Each player's army is divided into \textit{wings} (Saxons) or \textit{nationalities} (Normans) for determining battle order. Norman nationalities are further divided into \textit{sections} of foot and knights.

\subsubsection{}

The Saxon Army initially consists of three wings - right, center and left - each commanded by a leader. The leader's command radius usually determines which units make up his wing.

\subsubsection{}

Wings are defined at the beginning of each individual Battle Turn. No Saxon leader can control less than 20\% (1/5) of the Saxon units available. A given leader can control any number of units within his command radius as long as no Saxon leader controls less than 20\% of the Saxon units.

\subsubsection{}

If the Saxons are reduced to less than three leaders, no leader can control less than 33\% (1/3) of the Saxon units. If a Saxon leader dies during a Battle Turn, his wing is what it originally was until its units come under the control of the remaining Saxon leaders. If the Saxons have no leaders, they are assumed to have only one wing.

\subsubsection{}

The Norman army has six sections: Norman Foot/Bowmen, Norman Knights, Breton Foot/Bowmen, Breton Knights, Franco-Flemish Foot/Bowmen, and Franco-Flemish Knights. William's personal guard is \textit{not} part of any section (see 4.38). Norman leaders do not define Norman sections.

\subsubsection{}

Leaders never affect the order of their units. An army can operate without leaders, albeit with reduced efficiency.