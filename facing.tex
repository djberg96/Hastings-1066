\section{FACING}
\hfill

GENERAL RULE:

After finishing all movement, all combat units must be \textit{faced} (oriented) in a specific direction. The facing of a unit affects its ability to attack and defend. Facing affects movement or reaction only when the unit is routed. Leader units are not subject to the facing rules.

\subsection{How Units Are Faced}

\subsubsection[Facing Direction]{} All combat units must be faced in a specific direction after they finish movement and maintain that facing until they move again. Facing does not use movement points. A unit that is eligible to move and does not can still change its facing.

\subsubsection[Combat Units]{} Combat units are faced so the top of the unit points to a vertex of its hex. Each unit has two \textit{frontal}, two \textit{flank} and two {rear} hexsides.

\subsubsection[Battle Order]{} Battle order never affects facing.

\subsection{Effects of Facing on Combat}

\subsubsection[Missile Armed Troops]{} Missile-armed units can fire only through their frontal hexsides,

\subsubsection[Melee]{} Units can melee only those enemy units that are in their frontal (controlled) hexsides. A unit must attack all enemy units in its ZOC, unless the enemy unit is being meleed by another friendly unit.

\subsubsection[Flank or Rear]{} Friendly units attacking the flank or rear hexsides of an enemy unit receive a bonus to their combat rating (either fire or melee), as given.

\begin{tabular}{ lcc }
  & \textbf{Flank} & \textbf{Rear} \\
  \hline
  Fire & +1 & +1 \\
  Melee & +1 & +2 \\
\end{tabular}

Thus, a Breton foot unit in Advance to Combat order meleeing a Saxon fyrd unit through the fyrd's rear hexside would have a melee rating of “6” rather than its printed rating of “4".

\subsubsection[Defensive Ratings]{} Defensive ratings do not change if the unit is attacked through its flank or rear. Thus, a housecarle unit in Shield Wall order meleed from the rear would defend with its Shield Wall strength of “8.”