\section{INITIAL DEPLOYMENT}

\subsection{Norman Deployment}

At the start of the game, the Normans place their troops in the hexes marked on the game map, as follows:

\textbf{BB} = Breton Bowmen; \textbf{BF} = Breton Foot; \textbf{BK} = Breton Knights; \textbf{NB} = Norman Bowmen; \textbf{NF} = Norman Foot; \textbf{NK} = Norman Knights; \textbf{FF} = Franco-Flemish Bowmen; \textbf{FF} = Franco-Flemish Foot; \textbf{FK} = Franco-Flemish Knights. \textbf{Alan} is placed with any Breton unit; \textbf{Odo} is placed with any Breton unit; and \textbf{Eustace} with any Franco-Flemish unit. \textbf{William} is placed with his personal guard. The units can face in any direction.

\subsection{Saxon Deployment}

The Saxon player sets aside five 4-4/2-6 Thegn units. He then puts all the remaining fyrd units into a cup and randomly draws 27 of them. These, together with the five Thegn units are used by the Saxon player during the game. The pieces are set up as follows:

\textbf{F/T} = Fyrd/Thegn; \textbf{HC} = Housecarl; \textbf{B} = Bowmen. \textbf{Harold is placed in hex 0514;} \textbf{Leofwine} in hex 0710; and \textbf{Gyrth} in 0721. The units can face in any direction.

\subsection{Variable Deployment (Optional)}

The players can, if both agree, ignore the previous two cases and set up their units using case 11.5 as a guideline. The Normans are not required to set up within two hexes of the edge of Senlac Hill, as long as all units are at least four hexes south of the hill.

\subsection{Saxon Reinforcements}

\subsubsection[First Assault Period]{} During the First Assault Period, the Saxon player receives twelve (12) Great Fyrd units chosen randomly from those not picked at the beginning of the game.

\begin{itemize}
  \item Two units arrive in each of the first four Battle Turns, one unit arrives in each of the last four Battle Turns.
  \item Reinforcing units arrive in hex 0116 during the friendly movement segment. If two units arrive, the second one to enter has already used a movement point before entering the map.
  \item Arriving reinforcements pay the cost of the first hex they enter. If hex 0116 is blocked, reinforcements can enter through the nearet unoccupied hex on the north map edge.
  \item Reinforcements are in Advance to Melee order until they come within a leader's command range (see 4.38).
\end{itemize}

\subsubsection[Remaining Units]{} The Saxon player receives the remaining twelve fyrd units between the Assault Periods (11.5). These can be immediately placed anywhere on Senlac Hill.

\subsection{Reforming Armies}

\subsubsection[First Assault Period]{} If the First Assault Period ends without a Norman victory (see 12.1), then all combat stops, routed and disrupted units are checked and the armies are removed from the map and reformed. The Normans reform first.

\subsubsection[Routed and Disrupted Units]{} All routed and disrupted units on the map at the end of the First Assault Period are automatically rallied for the next period as long as at least one friendly leader remains on the map. If no friendly leader is present, they are eliminated.

\subsubsection[Reforming Area]{} At the end of the First Assault Period, all units are examined to see if they can trace a path of hexes free of enemy units or zones of control to their reforming area or to any friendly unit that can trace such a path. Units that cannot trace such a path are eliminated. For this purpose, friendly units negate enemy ZOC. Any ambiguous cases are decided in favor of the threatened unit.

\subsubsection[Norman Reform]{} The Normans reform their line as desired, at least four hexes south of Senlac Hill. Their flanks cannot extend eastward of the xx06 row or westwar of the xx26 row.

\subsubsection[Norman Rearrangement]{} The Norman player can rearrange the \textit{location} of his sections at the beginning of the Second Assault Period. He can't intermix nationalities (e.g. Bretons with Normans) or types (foot with knights).

\subsubsection[Saxon Reform]{} The Saxon units reform in any manner they wish anywhere on Senlac Hill (the highest two elevation levels of their original deployment area). The last twelve Saxon reinforcements are placed at this time.

\subsubsection[Saxon Reduction]{} The Saxons can reduce the number of wings in their army if they meets the limits of 4.12 and 4.13.
