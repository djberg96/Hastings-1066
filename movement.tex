\section{Movement}

GENERAL RULE:

During the movement segment of a Battle Turn, the phasing player can move as many or as few of his units as he wishes, in any direction or combination of directions. Each unit has a \textit{movement allowance} (5.12) that can be restricted by its Battle Order. Normally, however, each unit can spend as many of its movement points (MPs) as desired within the limits of its movement allowance.

PROCEDURE:

Units are moved one at a time, tracing a path of contiguous hexes through the hex grid. As a unit enters a hex or crosses a hexside, it must pay part of its movement allowance. These costs are listed on the Terrain Effects Chart on the map.

\subsection{How to Move}

\subsubsection[Movement Calculation]{} Movement is calculated in terms of hexes. Basically, each unit spends one movement point for each hex that it enters. Some hexes and hexsides cost additional movement points to enter or cross, as listed on the Terrain Effects Chart.

\subsubsection[Movement Costs]{} The movement costs for all units are as follows:

\begin{center}
  \begin{tabular}{ |c|c| }
    \hline
    \textbf{Unit} & \textbf{Movement Allowance} \\
    \hline
    Knights (Cavalry) & 4 MP (normal) \\
    & 6 MP (charging) \\
    Leader & 6 MP \\
    All Others (Foot, Bowmen, etc) & 3 MP \\
    \hline
  \end{tabular}
\end{center}

These are the \textit{maximum} number of movement points the unit can use within a given movement segment. \textit{See 4.3 for battle order restrictions on movement allowances.}

\subsubsection[Movement Limits]{} Units can be moved only once in a movement segment. However, certain movement (such as rout, pursuit and reaction) is not considered movement under this section. Rout (9.23) and pursuit (9.4) do not use movement points, nor are they restricted by battle order. Reaction does not use movement points, but \textit{is} restricted by battle order.

\subsubsection[Movement Point Use]{} No unit is required to use all (or any) of its allotted movement points. However, unused movement points cannot be transferred from unit to unit, nor can they be saved for the next movement segment.

\subsubsection[Movement and Facing]{} Units can move into any hex for which they have the movement points available (5.21). The \textit{facing} of a unit (6.0) does not affect its movement (however, see 6.11). Leaders have no effect on movement.

\subsection{Restrictions on Movement}

\subsubsection[Moving Through Units]{} All units can always move through other friendly units. Combat units cannot end their movement in the same hex as another combat unit. Leaders can end their movement in the same hex with other friendly units. Friendly units cannot enter an enemy-occupied hex.

\subsubsection[Movement and Combat]{} There is no combat during movement, and enemy units cannot move in a friendly movement segment.

\subsubsection[No Skipping]{} Movement from hex to hex must be consecutive; units cannot "skip" hexes.

\subsubsection[Movement During Fire and Melee]{} There is usually no movement during the fire or melee segments. However, certain combat results call for a rout or pursuit, in which case the affected units are moved according to rules 9.23 and 9.4.

\subsubsection[Battle Order Limits]{} Certain battle orders limit the movement allowances of units that adopt them (see 4.3).

\subsection{The Effects of Terrain on Movement}

\subsubsection[Hex Costs]{} Each hex - and certain types of hexsides - have a specific movement point cost to enter or cross (see the Terrain Effects Chart).

\subsubsection[Terrain Color]{} The color of a hex on the game map represents its approximate elevation; the darker hued the hex, the higher the elevation. Terrain features are represented by symbols in the hex or along the hexside. For example, the slopes hexsides that separate certain elevations represent relatively sharp increases in height. A given hex is always the height of the highest elevation in that hex.

\subsubsection[Uphill]{} The term "uphill" means any movement that either goes from a lower elevation to a higher elevation \textit{or} any movement that takes a unit within \textit{two} hexes of a higher elevation moving in the direction of the higher elevation. There are two exceptions:

\begin{itemize}
  \item Movement within two hexes of the hill in hex 1419 or the "top" of Senlac Hill (275 foot elevation) does not count as uphill. Units must actually move \textit{into} these hexes or the upslope hexes of the hill to be moving uphill; and
  \item Units moving laterally to or away from a higher elevation are not moving uphill, regardless of the hex they are in. Thus, a unit moving from 0918 to 0919 or 0920 is not moving uphill.
\end{itemize}

\subsubsection[Downhill]{} "Downhill" is the reverse of uphill; using the precepts of 5.33, but with the direction reversed.

\subsubsection[Norman Knights]{} Each time a Norman knight unit tries to cross a ridge while moving in \textit{any} direction, the knight unit undergoes an immediate morale check in the hex preceding the ridge, treating all routs as disruptions. If the knight unit does not become disrupted, its movement ends. It remains in the hex until rallied or moved in a later segment. If undisrupted, it continues its move normally.

\subsubsection[Knights and Marsh]{} Each time a knight unit enters a marsh hex it must undergo an immediate morale check as detailed in 5.35. Again, all routs are treated as disruptions.

\subsubsection[The Road]{} The road has no effect on movement or combat.

\subsection{Reaction}

\subsubsection[Reaction Defined]{} Reaction is a form of movement used only by the non-phasing player. Reaction movement occurs only in the reaction segment of the other player's phase.

\subsubsection[How Reaction Works]{} Units capable of reaction (see 5.44) can, if desired, move \textit{one} hex away from an enemy combat unit during a friendly reaction segment. Such movement does not use or cost movement points; moreover, terrain has no effect on such movement.

\subsubsection[Reaction and ZOCs]{} Units using reaction cannot move into enemy zones of control (ZOC).

\subsubsection[Eligible Units]{} Only eligible units can use reaction. All units are eligible except the following:

\begin{enumerate}
  \item Saxon foot units in the ZOC of an enemy (mounted) Knight.
  \item Any unit in the following order: Shield Wall, Attack \& Pursue, or Charge.
  \item Any unit not in the command range of a friendly leader.
\end{enumerate}

\subsubsection[Morale Check]{} Any unit that uses reaction undergoes a morale check upon completing the movement.

\subsubsection[Missile Fire]{} A unit using reaction cannot use missile fire in the following defensive missile fire segment, unless the moving unit is a bowman unit. Bowman units can both react and fire.

\subsection{Terrain Effects Chart} \textbf{see map}

\subsection{Stacking}

\subsubsection[Limits]{} Only one combat unit can end a movement, fire or melee segment in any one hex. Leaders can always stack with friendly units. Informational markers can also be placed in hexes with other counters.

\subsubsection[Moving Through Units]{} Friendly units of all types can freely pass through each other during movement, reaction, rout, pursuit, etc.

\subsubsection[Knights and Stacking]{} A knight unit cannot be moved into an occupied hex if such movement would cause a morale check (see 5.35 and 5.36). This is an exception to 5.62.

\subsection{Zones of Control}

All combat units that are not disrupted or routed exert a zone of control (ZOC) through their frontal hexsides into adjacent hexes. Such hexes are \textit{controlled} by the units exerting the ZOC. Leaders and routed units do not exert ZOC's. ZOC's are not affected by terrain. ZOC's themselves affect movement and combat.

\subsubsection[Stopping Movement]{} A unit must stop whenever it enters an enemy-controlled hex. It can move no further in that movement segment. A combat unit cannot enter an enemy ZOC that is occupied by another friendly combat unit.

\subsubsection[Leaving a ZOC]{} Friendly units beginning a movement segment in an enemy ZOC can leave the ZOC as long as they do not move directly into another enemy-controlled hex. Such units may eventually enter an enemy-controlled hex in the same segment as long as the movement is not directly from one enemy ZOC to another.

\subsubsection[Bowmen and ZOCs]{} Bowmen untis can never voluntarily enter an enemy ZOC. Such units in an enemy ZOC at the start of a friendly movement segment must move out of the ZOC if they can. There is no penalty if they are unable to do so.

\subsubsection[MP and ZOCs]{} There is no additional movement point cost to enter or leave enemy ZOC.

\subsubsection[Competing ZOCs]{} If both friendly and enemy units exert a ZOC into the same hex, both units control the hex. There is no additional effect for having more than one unit control a hex.

\subsubsection[Friendly ZOCs]{} Friendly ZOC's never affect friendly units.

\subsubsection[ZOCs and Combat]{} The effects of ZOC's on combat are given in 6.24 and 8.11.