\section{Movement}

GENERAL RULE:

During the movement segment of a Battle Turn, the phasing player can move as many or as few of his units as he wishes, in any direction or combination of directions. Each unit has a \textit{movement allowance} (5.12) that can be restricted by its Battle Order. Normally, however, each unit can spend as many of its movement points (MPs) as desired within the limits of its movement allowance.

PROCEDURE:

Units are moved one at a time, tracing a path of contiguous hexes through the hex grid. As a unit enters a hex or crosses a hexside, it must pay part of its movement allowance. These costs are listed on the Terrain Effects Chart on the map.

\subsection{How to Move}

\subsubsection[Movement Calculation]{} Movement is calculated in terms of hexes. Basically, each unit spends one movement point for each hex that it enters. Some hexes and hexsides cost additional movement points to enter or cross, as listed on the Terrain Effects Chart.

\subsubsection[Movement Costs]{} The movement costs for all units are as follows:

\begin{center}
  \begin{tabular}{ |c|c| }
    \hline
    \textbf{Unit} & \textbf{Movement Allowance} \\
    \hline
    Knights (Cavalry) & 4 MP (normal) \\
    & 6 MP (charging) \\
    Leader & 6 MP \\
    All Others (Foot, Bowmen, etc) & 3 MP \\
    \hline
  \end{tabular}
\end{center}

These are the \textit{maximum} number of movement points the unit can use within a given movement segment. \textit{See 4.3 for battle order restrictions on movement allowances.}

\subsubsection[Movement Limits]{} Units can be moved only once in a movement segment. However, certain movement (such as rout, pursuit and reaction) is not considered movement under this section. Rout (9.23) and pursuit (9.4) do not use movement points, nor are they restricted by battle order. Reaction does not use movement points, but \textit{is} restricted by battle order.

\subsubsection[Movement Point Use]{} No unit is required to use all (or any) of its allotted movement points. However, unused movement points cannot be transferred from unit to unit, nor can they be saved for the next movement segment.

\subsubsection[Movement and Facing]{} Units can move into any hex for which they have the movement points available (5.21). The \textit{facing} of a unit (6.0) does not affect its movement (however, see 6.11). Leaders have no effect on movement.