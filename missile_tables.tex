\documentclass{article}
\usepackage{xcolor,colortbl}
\usepackage{graphicx}

\definecolor{BabyBlue}{rgb}{0.54, 0.81, 0.94}
\thispagestyle{empty}

\begin{document}
  \def\arraystretch{1.2}
  \noindent\makebox[\textwidth]{
    \begin{tabular}{ |lcc| }
      \hline
      \rowcolor{BabyBlue}
      \multicolumn{3}{|c|}{\textbf{[7.45] MISSILE SUPPLY TABLE}} \\
      \textbf{Assault Period} & \multicolumn{2}{c|}{\textbf{Fire Segments Allowed}} \\
      & Norman Bowmen & Saxon Jav/Sps \\
      \hline
      First & 6 & 4 \\
      Second & 3 & 2 \\
      \hline
    \end{tabular}
  }

  \vspace*{2em}

  \noindent\makebox[\textwidth]{
    {
      \def\arraystretch{1.2}
      \begin{tabular}{ |lcccc| }
        \hline
        \rowcolor{BabyBlue}
        \multicolumn{5}{|c|}{\textbf{[7.51] MISSILE FIRE MATRIX}} \\
        & & \multicolumn{3}{c|}{ \textbf{Hex Range}} \\
        \hline
        Weapon & Max. Range & 3 & 2 & 1* \\
        \hline
        Bow & 3 hexes & 3 & 4 & 5 \\
        Javelin & 2 hexes & - & 2 & 3 \\
        Sling & 3 hexes & 1 & 1 & 2 \\
        \hline
        \multicolumn{5}{|c|}{Number is missile fire strength of unit.} \\
        \multicolumn{5}{|c|}{Bowmen may not enter enemy ZOC.} \\
        \multicolumn{5}{|c|}{Bowmen in enemy ZOC must leave ZOC.} \\
        \hline
      \end{tabular}
    }
  }

  \vspace*{2em}

  {\def\arraystretch{1.2}
    \resizebox{\columnwidth}{!}{\begin{tabular}{ |ccccccccccc| }
        \hline
        \rowcolor{BabyBlue}
        \multicolumn{11}{|c|}{\textbf{[7.52] MISSILE FIRE COMBAT RESULTS TABLE}} \\
        \multicolumn{11}{|c|}{ \textbf{Fire Ratio} } \\
        \hline& & & & & & & & & &\\[-2.5ex]
        Die Roll & 1-4 & 1-3 & 1-2 & 1-1.5 & 1-1 & 1.5-1 & 2-1 & 3-1 & 4-1 & 5-1 \\
        \hline
        1 & -- & -- & -- & -- & -- & M & M & M & D & D \\
        \rowcolor{BabyBlue}
        2 & -- & -- & -- & -- & M & M & M & D & D & D \\
        3 & -- & -- & -- & M & M & M & D & D & D & 1 \\
        \rowcolor{BabyBlue}
        4 & -- & -- & M & M & M & D & D & D & 1 & 1 \\
        5 & -- & M & M & M & D & D & D & 1 & 1 & 1 \\
        \rowcolor{BabyBlue}
        6 & M & M & M & D & D & D & 1 & 1 & 1 & 1 \\
        \hline
    \end{tabular}}
  }
\end{document}