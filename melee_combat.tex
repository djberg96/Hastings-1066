\section{MELEE COMBAT}
\hfill

GENERAL RULE:

Melee takes place in the melee segment of the turn. The phasing player is the attacker; the other is the defender. Units can only attack through their frontal hexsides, i.e only those enemy units in hexes they control. Melee is resolved by comparing the melee strengths of the opposing units on the Melee Results Table. Leaders and Terrain can affect Melee. Losses and retreats are taken after each melee is resolved.

PROCEDURE:

The attacker indicates which of his units will attack which enemy units. He then totals the melee strengths of his attacking units (noting battle order) and subtracts the defensive strength of the enemy unit, adjusting for terrain and leaders. The resultant number is the \textit{melee differential}, which can be a plus or minus number (or zero). (Thus, 4 points meleeing 6 points would have a differential of -2). Noting the differential, the attacker rolls one die, consults the Melee Results Table and cross-references that die roll on the table with the differential. Any results are applied immediately.

\subsection{Which Units Can Melee}

\subsubsection[ZoC]{} A unit can melee enemy units that are in its zone of control. A unit cannot melee through its flank or rear hexsides.

\subsubsection[Mandatory Attacks]{} A friendly unit \textit{must} melee any enemy unit in its ZOC if that enemy unit also exerts a ZOC on the friendly unit. Otherwise attacking is voluntary. If a friendly unit has two enemy units in its ZOC, but only one unit exerts a ZOC on the friendly unit, the friendly unit can attack \textit{only} the enemy unit exerting the ZOC.

\subsubsection[Disrupted Units]{} Any combat unit that is not disrupted or routed can melee. Disrupted/routed units cannot attack (even under 8.12), but can only defend against melee attacks.

\subsubsection[Only One Melee]{} A unit can only participate in one melee per melee segment. No enemy unit can be meleed more than once in any melee segment.