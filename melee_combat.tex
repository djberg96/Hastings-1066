\section{MELEE COMBAT}
\hfill

GENERAL RULE:

Melee takes place in the melee segment of the turn. The phasing player is the attacker; the other is the defender. Units can only attack through their frontal hexsides, i.e only those enemy units in hexes they control. Melee is resolved by comparing the melee strengths of the opposing units on the Melee Results Table. Leaders and Terrain can affect Melee. Losses and retreats are taken after each melee is resolved.

PROCEDURE:

The attacker indicates which of his units will attack which enemy units. He then totals the melee strengths of his attacking units (noting battle order) and subtracts the defensive strength of the enemy unit, adjusting for terrain and leaders. The resultant number is the \textit{melee differential}, which can be a plus or minus number (or zero). (Thus, 4 points meleeing 6 points would have a differential of -2). Noting the differential, the attacker rolls one die, consults the Melee Results Table and cross-references that die roll on the table with the differential. Any results are applied immediately.

\subsection{Which Units Can Melee}

\subsubsection[ZoC]{} A unit can melee enemy units that are in its zone of control. A unit cannot melee through its flank or rear hexsides.

\subsubsection[Mandatory Attacks]{} A friendly unit \textit{must} melee any enemy unit in its ZOC if that enemy unit also exerts a ZOC on the friendly unit. Otherwise attacking is voluntary. If a friendly unit has two enemy units in its ZOC, but only one unit exerts a ZOC on the friendly unit, the friendly unit can attack \textit{only} the enemy unit exerting the ZOC.

\subsubsection[Disrupted Units]{} Any combat unit that is not disrupted or routed can melee. Disrupted/routed units cannot attack (even under 8.12), but can only defend against melee attacks.

\subsubsection[Only One Melee]{} A unit can only participate in one melee per melee segment. No enemy unit can be meleed more than once in any melee segment.

\subsection{Multiple-Unit and Multi-Hex Combat}

\subsubsection[Multiple Units]{} More than one friendly unit can attack a given enemy unit, as long as all facing rules are observed. The strengths of all attacking units are combined into a single total.

\subsubsection[One Unit Versus Many]{} A single friendly unit can attack as many two enemy units (one through each frontal hexside), in which case the defensive strengths of both enemy units are totalled. (\textit{Exception: see 8.12}.)

\subsubsection[Attacks and ZOCs]{} Attacks can involve a variety of attacking and defending units in different hexes. However, remember that each attacking unit must exert a ZOC on any enemy unit it attacks. Moreover, see 8.12.

\subsubsection[No Split Strength]{} No single unit can split its melee strength to attack more than one enemy unit in more than one combat.

\subsubsection[Different Results]{} When more than one unit attacks a single defending unit, they are \textit{all} affected by an \textbf{M} or \textbf{D} result, but only one attacking unit takes a step loss on a "1" result.

\subsection{Melee Resolution}

\subsubsection[Procedure]{} Melee is resolved as per the instructions in the \textit{Procedure} for melee combat. Melees are resolved in any order the phasing player wishes, with all results being taken immediately.

\subsubsection[Differentials]{} All melees conducted at a differential greater than +6 are considered to be +6. All attacks at -6 or worse are an automatic one step loss for the attacker and have no effect on the defender.

\subsubsection[Results]{} See 9.2 for an explanation of combat results.

\subsection{The Effects of Terrain on Melee}

Terrain affects combat in several ways, all of which are listed on the Terrain Effects Chart (5.5). The adjustments for attacking across a ridge hexside apply only if the attack is made through the hexside - not if the unit moves through the ridge to get to the melee. Terrain never prohibits melee.

\subsection{The Effects of Leaders on Melee}

If a leader is stacked in a hex with a unit that is either attacking or defending in melee, add one to the strength of the unit. If the leader is either Harold or William, add \textit{two} to the strength.